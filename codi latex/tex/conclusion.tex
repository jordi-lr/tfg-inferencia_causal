\documentclass[../main.tex]{subfiles}

\begin{document} 

\chapter{Conclusions} \label{ch:conclu}

El treball tenia per objectiu apropar-se a la inferència causal per poder abordar problemàtiques sobre l’efecte heterogeni del tractament; és a dir, explorar com pot variar la resposta a la intervenció segons les característiques de cada cas. Per a això s’ha construït un marc teòric que pretén entendre les bases de la inferència causal i la manera d’enfocar els efectes heterogenis, identificant les eines disponibles i aprofundint en els meta-learners (S-, T- i X-learner). Finalment, s’ha aplicat tot aquest coneixement a dades reals per estudiar com varia l’efecte de dos tractaments en l’evolució d’embarassos amb risc de SGA; s’han analitzat vuit outcomes (tres continus i cinc binaris) comparant control i dos tractaments diferents.\par
De tot aquest recorregut teòric culminat amb un cas pràctic se’n poden extreure diverses conclusions en diferents àmbits.\par
D’entrada cal remarcar que la inferència causal aborda una qüestió bàsica per a l’estadística, ja que vol respondre a preguntes del tipus «per què passa X?». La dificultat essencial rau en estimar la diferència entre dos estats (tractat i no tractat) sense poder observar-ne simultàniament tots dos. Aquesta impossibilitat impedeix conèixer el rendiment real i l’error exacte dels mètodes, i exigeix que es compleixin tres assumpcions bàsiques, consistència, ignorabilitat i positivitat, i que s’entengui com flueix la causalitat a través de les variables observades i les ocultes (en aquest sentit els DAG poden ser de gran utilitat). \par
Amb l’avenç de la potència computacional i l’auge del machine learning, que aprofita aquesta potència per predir resultats, han aparegut diverses eines que l’apliquen a la causalitat. Al treball s’han presentat breument diferents estratègies, però s’ha incidit especialment en els tres meta-learners descrits a \citealp{kunzel2019}. Tots tres parteixen de la idea d’utilitzar les dades per entrenar un o diversos models (en funció del meta-learner) que prediuen la resposta; és a dir, busquen estimar els resultats potencials en els dos estats que es volen comparar.\par
A la part pràctica s'han utilitzat aquests tres mètodes per estudiar dades reals. Aquesta aplicació ha permès entendre les implicacions tractades a la secció més teòrica i, alhora, veure el rendiment dels meta-learners per fer-se una idea de com funcionen i quan és millor cadascun:
\begin{itemize}
    \item \textbf{S-learner}: És, amb diferència, el més senzill, només cal entrenar un model, cosa que el fa el menys costós computacionalment. Però aquesta simplicitat comporta resultats pitjors; tal com ja advertia la teoria, el model tendeix a un biaix cap al zero. En el nostre cas, tot i que els grups estaven equilibrats, sempre ha obtingut valors molt propers a zero, llunyans als dels altres dos mètodes i amb una variància força menor.
    \item \textbf{T-learner}: Afegeix un grau de complexitat amb un model addicional, fet que li permet capturar relacions més complexes a canvi d’un cost computacional lleugerament superior. La teoria remarca el seu mal funcionament amb grups desequilibrats, però com que les nostres dades estaven ben equilibrades, això n’ha afavorit el rendiment. Ha estat el meta-learner amb més variància entre els ITEs predits, indicant una possible inestabilitat, però l’estimació de l’efecte mitjà del tractament ha estat molt similar a la de l’X-learner i a l’estudi original.
    \item \textbf{X-learner}: És el més complex dels tres, ja que utilitza quatre models més un altre per estimar el propensity score; això dobla el cost computacional del T-learner. Tanmateix, aquest cost ve acompanyat d’un augment de la robustesa: les prediccions són més fiables i amb menys variància que el T-learner. Aquesta fiabilitat extra ens ha fet decantar-nos per ell a l’hora d’estudiar els HTE. A més, encara que el nostre conjunt fos equilibrat, la seva arquitectura (que incorpora el propensity score) el fa especialment adequat per a situacions desequilibrades.
\end{itemize}

Un gran avantatge dels meta-learners és que accepten tota mena de variables de resposta. Malgrat que, excepte el S-learner, estan limitats a tractaments binaris.\par
Quant a la pràctica, disposar de la distribució completa dels ITEs ens ha permès fer inferència sobre els efectes heterogenis utilitzant tècniques com el bootstrap i els mètodes lineals. D’aquesta manera s’ha pogut determinar com varia l’efecte dels tractaments segons el perfil maternal.\par
De l’estudi pràctic dels HTE s’observa que no totes les mares responen igual a la intervenció i que certes característiques influeixen de manera especial en aquestes diferències. També es constata que, per a cada outcome i tractament, emergeixen variables diferents com a causants de l’heterogeneïtat. Aquest fet suposa un argument a favor de la medicina personalitzada, ja que aporta evidència que els tractaments no tenen els mateixos efectes en tots els pacients. Caldria, evidentment, recopilar més dades per disposar d’estimacions més robustes i, eventualment, construir prediccions individuals acurades; però, fins i tot com a estudi exploratori, el treball ja posa de manifest la necessitat d’analitzar en profunditat el perfil de les mares que més es beneficien d’aquestes intervencions.\par
Alhora de recomanar un tractament a una nova pacient ls diferents CATE per subgrups ofereixen una primera estimació de com pot variar l’efecte d’un tractament segons la categoria on se situï la gestant. A més, els meta‑learners entrenats en aquest estudi permeten projectar l'ITE d’una nova pacient, tal com s’explica a la secció \ref{sec:concl_parct}. Així, el model pot predir l’efecte esperat sobre qualsevol gestant amb característiques comparables a les de la mostra i, en funció del resultat, recomanar o no el tractament. Ara bé, la validesa d’aquestes prediccions queda restringida a poblacions semblants a la de l’assaig original, motiu pel qual l’ampliació de la mostra és indispensable per millorar la generalització.\par
En definitiva, aquest treball no només introdueix una metodologia potent per estudiar causalitat en contextos complexos, sinó que també destaca la necessitat de considerar la variabilitat individual com a element clau en la presa de decisions mèdiques.

    
\end{document}